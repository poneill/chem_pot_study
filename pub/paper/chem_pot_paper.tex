\documentclass{article}
\usepackage{amsmath,graphicx,setspace,fullpage,subfigure,authblk}
\begin{document}
\title{Heterogeneous Langmuir Kinetics Improves Self-Consistency of Transcription Factor Binding Models}
\author[1]{Patrick O'Neill}
\author[1]{Ivan Erill}
\affil[1]{Department of Biological Sciences, UMBC}
\maketitle{}
\section{Introduction}

\doublespacing Transcriptional regulation is the most ubiquitous and
efficient mechanism for control of gene expression in biological
systems.  Despite the prevalence and importance of this topic, many
fundamental questions regarding the physiology and evolution of
transcriptional regulatory systems remain open.  Although several
specific systems have been studied in great detail
(\textit{e.g.} \textit{lac},\textit{ara},$\lambda$)\cite{riggs70},\cite{hamilton88},\cite{ptashne67}
within bacterial models, the overwhelming number of distinct
transcription factors (TFs) present within a typical bacterial cell
($\sim 200$) motivates the search for models of transcriptional
regulation with general applicability.

Models of TF-DNA interaction are broadly divided into two classes by
their theoretical frameworks: probability and thermodynamics.
Probabilistic models generally consider the genome as a mixture model
with two components, one corresponding to the genomic background and
the other corresponding to the set of functional binding sites.  Given
a putative site, one may compute the log-odds ratio for the components
of the mixtures and interpret the biological function of the site in
relation to the probability of the site belonging to the binding site
component.  Such models usually suffer from a high number of false
positives\cite{erill09}.  In the thermodynamic approach, one models
the TF as a function from nucleotide sequences to free energies of
binding.  Good functions are those which separate true sites from the
genomic background, assigning lower free energies of binding to the
former; in practice, the energy function is assumed to be
well-approximated by a weight matrix parametrized by collections of
experimentally verified sites.  Given the binding energies for every
potential binding site in the genome, the expected occupancies
(i.e. probability that the site is bound) can be approximated to first
order by a Boltzmann distribution, or by Fermi-Dirac statistics if TF
copy number is to be taken into account \cite{gerland02}.

In this study, we propose to address two common assumptions made in
most existing models of TF-DNA interaction.  The first assumption that
the copy number is small, in a sense that will be clarified below.
The second assumption is that the genomic background is
well-approximated by a mononucleotide model.  Both assumptions were
originally granted for the purposes of analytical tractability, but in
an era of ubiquitous sequence data and cheap computing power it has
become possible to re-examine these simplifications and compare their
theoretical entailments to the available data.

\section{Binding Energy Model}
\subsection{Overview}
We consider the case of $k$ TF copies binding to a circular chromosome
of length $G$ containing $G$ distinct (but potentially overlapping)
possible binding sites where site $s_i$ has a given free energy
$\epsilon_i $ associated with the binding event.  At any time, each
site may be either \textit{bound} or \textit{unbound} by a TF copy.
Sites may be bound in either of the \textit{specific} or
\textit{non-specific} modes.  We assume that there is a single, global
value of free energy $\epsilon_{ns}$ associated with non-specific
binding to any site.

% , driven respectively by hydrogen bonding
% between specific nucleotides of the site and amino acid residues of
% the DNA-binding domain of the protein, and generic electrostatic
% interactions.

We further assume that steric exclusion prevents more than one copy
from binding to a given site simultaneously.  Since each site is
offset by one nucleotide from the site preceding it, it is possible
that a protein bound to a given site will interfere with other copies
binding at neighboring sites.  So long as sites with high
probabilities of occupancy are not extremely co-localized in the
chromosome, however, interference between neighboring sites can be
neglected.  We therefore assume that the binding state of one site
does not affect the occupancy of any other state, either through
steric exclusion or cooperative binding.


\subsection{Inference of binding energies from sequence data}
In general, tolerably accurate estimations of the collection of
binding energies $\vec{\epsilon}$ can be read off of the sequence
data, and we will consider them as given throughout.  We use the TRAP
model \cite{roider07}, which proposes the relation:

\begin{equation}
  \label{eq:trap_init}
  p(s) = \frac{R_0e^{-\beta E(s)}}{1 + R_0e^{-\beta E(s)}},
\end{equation}

where
\begin{equation}
  \label{eq:trap}
  \beta E(s) = \frac{1}{\lambda} \mathrm{HI}(s),
\end{equation}

 $\mathrm{HI}(s)$ is the Heterology Index:

\begin{equation}
  \label{eq:bvh}
  \mathrm{HI}(s) = \displaystyle\sum_{i=1}^W\ln(\frac{n_{i,\max}+1}{n_{i}+1})
\end{equation}
as defined in \cite{berg87}.  (In \cite{roider07} the BvH
function is modified slightly to incorporate a correction for
background mononucleotide frequencies.  Since the mononucleotide
frequencies in \textit{Escherichia coli} are essentially uniform, we
neglect that term here.)  The TRAP scoring function is parametrized by
$\lambda$, the mismatch scaling parameter, and $R_0 = K(s_0)[TF]$ for
$K(s_0)$ the association constant of the consensus sequence $s_0$.

Equation \ref{eq:trap_init} can be written as:

\begin{equation}
\label{eq:bindingprob}
  P(s) = \frac{1}{1+e^{-\beta(E(s) - \frac{\ln(R_0)}{\beta})}}
\end{equation}

so that we can interpret:

\begin{equation}
  \label{eq:bindingenergy}
  \epsilon(s) \equiv (E(s) - \frac{\ln(R_0)}{\beta}) = \frac{\frac{1}{\lambda}\mathrm{HI(s)} - \ln R_0}{\beta}
\end{equation}

directly as a free energy of binding in units of $k_BT$.  The parameter values $(\lambda = .7, \ln R_0 = 0.585 w - 5.66)$ for $w$ the width of the binding site were estimated from empirical data in \cite{manke08}.

\subsection{Strandedness and Non-specific binding}

Two secondary issues slightly complicate the modeling of a TF binding
event: the fact that chromosomal DNA is double-stranded and the fact
that TFs may bind either specifically or non-specifically.

\subsubsection{Strandedness}
Double-stranded DNA may be bound by a TF in either of two
orientations, which we will refer to as the \textit{forward} and
\textit{backward} orientations.  Since binding in the two orientations
is mutually exclusive and jointly exhaustive given the binding event,
we may write the occupancy of the site as:
\begin{equation}
P(s) = P(s_f) + P(s_b)
\end{equation}
where $s_f$ and $s_b$ represent the binding event in the forward and backward orientations, respectively.  After substitution with Eq. \ref{eq:bindingprob}, we obtain:
$$e^{-\beta \epsilon_s} = e^{-\beta \epsilon_f} + e^{-\beta \epsilon_b}$$
where $\epsilon_f$ and $\epsilon_b$ are the free energies of binding in
each orientation.  This yields the specific free energy
$\epsilon_s$ which summarizes the specific binding in both strands:

\begin{equation}
\label{eq:doublestrandedbinding}
\epsilon_s = \frac{-\ln(e^{-\beta \epsilon_f} + e^{-\beta \epsilon_b})}{\beta}.
\end{equation}

If the binding site is asymmetric, then the contribution from one
orientation (say, $\epsilon_f)$ will predominate since $\epsilon_f \ll
\epsilon_b$, and so $\epsilon_s \approx \epsilon_f$.  If the binding
site is symmetric, however, then we will have $\epsilon_f \approx
\epsilon_b$, and $\epsilon_s \approx \epsilon_f - \frac{\ln
  2}{\beta}$.  Intriguingly, subtracted term is precisely Landauer's
limit, the minimum amount of energy required to erase one bit of
information  \cite{landauer61}. It is suggestive to interpret this
difference as the gain in free energy obtained by neglecting to track
the distinction between binding in the forward or reverse
orientations, as is physiologically reasonable for the homomultimeric
proteins which constitute the majority of prokaryotic transcription
factors.  Throughout, we will consider the free energy of specific
binding to a given site to be determined by
Eq. \ref{eq:doublestrandedbinding}, where $\epsilon_f$ is computed via
Eq. \ref{eq:bindingenergy} for the site in 5'-3' orientation and
$\epsilon_b$ for its reverse complement.

\subsubsection{Non-specific binding}
Transcription factors bind DNA in specific and non-specific modes.
The existence of non-specific binding was predicted from theoretical
considerations \cite{kaohuang77} and verified in many systems through
a variety of experimental techniques \cite{revzin90}.  Given various
estimates of non-specific binding of
$-7\frac{\mathrm{kcal}}{\textrm{mol}}$ for Lac repressor
\cite{wang77}, $-5.7\frac{\mathrm{kcal}}{\textrm{mol}}$ for MetJ
\cite{augustus10}, and $-5$ to $-7 \frac{\mathrm{kcal}}{\textrm{mol}}$
for CRP \cite{takahashi79}, $-4\frac{\mathrm{kcal}}{\textrm{mol}}$ and
$-4.2 \frac{\mathrm{kcal}}{\textrm{mol}}$ respectively for $\lambda$
cI and Cro \cite{bakk2004}, a global value $\epsilon_{ns}$ on the
order of $-5\frac{\mathrm{kcal}}{\textrm{mol}}\approx -8 k_BT$ at
physiological temperatures is assumed hereafter.

The total free energy of binding for a site is then given by:
\begin{equation}
  \epsilon = \frac{-\ln(e^{-\beta \epsilon_s} + e^{-\beta \epsilon_{ns}})}{\beta}.\label{eq:totalbindingenergy}
\end{equation}

  The total energy $\epsilon$ is effectively linear in $\epsilon_s$ for $\epsilon_s < \epsilon_{ns}$, and assumes the value $\epsilon_{ns}$ otherwise, in agreement with free energy profiles reported in \cite{maerkl2007}.

  A final complication arises from the fact that $\epsilon_{ns}$ is
  given in absolute terms, whereas values of $\epsilon_s$ are offset
  by an additive constant so that the specific binding energy for the
  consensus sequence $\epsilon_s(s_0)$ is defined to be 0.  To
  reconcile the two scales, we assume that the contribution of the
  specific binding is on the order of $2k_BT w$, an optimal search
  criterion derived from first principles \cite{gerland02} and
  essentially in agreement with experimental data \cite{bakk2004}.
\section{Deterministic Site Occupancy Model}
For each site we will have a reaction of the form:
\begin{equation}
  \label{eq:s_reaction}
  S_i + TF \overset{k^+_i}{\underset{k^-_i}{\rightleftharpoons}} C_i,
\end{equation}
\textit{i.e.} the $i$th site and a TF can enter into a reversible
complex at a site-specific rate.  Although binding energies are
usually reported in terms of $\Delta G$, it is often more convenient
to work with dissociation constants.  From the familiar Arrhenius law:

\begin{equation}\label{eq:kd}
  \frac{k^-_i}{k^+_i} = K_{di} = e^{\frac{\Delta G_i}{k_BT}}
\end{equation}
we may write down a system of differential equations:

\begin{equation}
  \label{eq:s_diff}
  \frac{1}{k_i^+}\frac{ds_i}{dt} = (1-s_i)p - K_{di}s_i
\end{equation}
where $s_i$ is the probability of occupancy of the $i$th site, and $p$
is the concentration of free protein.

At equilibrium ($\frac{ds_i}{dt} = 0$), we may solve for $s_i$ as a
function of free protein:
\begin{equation}
  \label{eq:s_equilibrium}
  s_i = \frac{p}{p + K_{di}}.
\end{equation}
The quantity of free protein, however, is not as easily determined.
There appears to be no alternative to solving the following equation:
\begin{equation}
  \label{eq:p_numeric}
  \overset{total\ protein}{\overbrace{k}} = \overset{bound\ protein}{\overbrace{\displaystyle\sum_i\frac{p}{p + K_{di}}}} + \overset{free\ protein}{\overbrace{p}}
\end{equation}
numerically.  Nevertheless, $p$ can be obtained quickly via standard
root-finding methods.

Lastly, Eq.\ref{eq:s_equilibrium} can be re-written using
Eq. \ref{eq:kd} and the identity $p = e^{\ln p}$, ultimately yielding:
\begin{equation}
  s_i = \frac{1}{1 + e^{\frac{\Delta G}{k_BT}  - \ln p}},
\end{equation}
which takes the form of the Fermi-Dirac distribution:
\begin{equation}
  \label{eq:s_fd}
  s_i = \frac{1}{1 + e^{\beta(\epsilon_i - \mu)}},
\end{equation}
where $\beta = \frac{1}{k_BT}$ is inverse temperature and $\mu$ is the
chemical potential.  This suggests the identity:

\begin{equation}
  \label{eq:mu_lnp}
  \mu = k_BT \ln p
\end{equation}

Equation \ref{eq:mu_lnp} is
especially interesting because it disagrees slightly with the
established approximation
\begin{equation}
  \label{eq:mu_approx}
  \hat{\mu} \approx k_BT (\ln k - \ln Z),
\end{equation}
given for example in \cite{gerland02} and (slightly modified) in
\cite{zhao09},where $Z=\sum_ie^{-\beta\epsilon_i}$ is the partition
function.  This discrepancy is revealing: the literature states that
$\mu$ should depend upon the total protein $k$, whereas we predict
that it should depend only upon the \textit{free} protein $p$.  This
invites us to consider the relationship between $k$ and $p$, which by
Eqs. (\ref{eq:mu_lnp},\ref{eq:mu_approx}) should be:

\begin{equation}
  \label{eq:p_vs_k}
  p \approx \frac{k}{Z}.
\end{equation}

\begin{equation}
  \frac{\partial\hat\mu}{\partial k}\propto \frac{\partial\ln k}{\partial k} = \frac{1}{k} < \frac{1}{p}
\end{equation}
Clearly, the approximation given in Eq. (\ref{eq:p_vs_k}) cannot
always hold.  For example, in the saturated limit $k\gg G$ we should
have $\frac{dp}{dk}\approx 1$: an additional protein will not bind at
all, having no free sites to bind to, and should contribute almost
entirely to the quantity of unbound protein. This limiting case
contradicts the approximation, which predicts $\frac{dp}{dk}\approx
\frac{1}{Z}$.  The relevance of this discrepancy will depend on the
behavior of the model in the physiologically realistic copy-number
regime, roughly $10^1 < k < 10^3$ \cite{ishihama08}.  

\section{Stochastic Site Occupancy Model}
In the previous section we presented a deterministic ODE model for the
time-evolution of the occupancy of each binding site.  Such a model
assumes that the copy number of the TF is large enough to justify mass
action kinetics.  Since this assumption may not always hold \textit{in
  vivo}, we consider in this section a stochastic model derived from
the chemical master equation for the heterogeneous Langmuir process.

Consider a system with $G$ binding sites and $k$ TF copies, having a
collection of states $X = \{x \in \{0,1\}^G | |x| \leq k\}$, where
$|x| = |x|_0$.  One may interpret $x$ as the state of the chromosome
in which a TF copy is bound at position $i$ iff $x_i = 1$, and $f(x) =
k - |x|$ copies of the TF are unbound in the cytosol.  Assuming some
fixed ordering on the set of states, the chemical master equation is
then:

\begin{equation}
  \label{eq:cme}
  \dot P(t) = A P(t)
\end{equation}

where $A_{yx} =
\begin{cases}
  f(y) k_i^+ & y \overset{i}{\rightarrow} x\\
  1 & y \overset{-i}{\rightarrow} x \\
  0 & \textrm{otherwise}
\end{cases}
$

and $y \overset{i}{\rightarrow} x$ (respectively, $y
\overset{-i}{\rightarrow} x$) denotes the relation that state $x$
is accessible from state $y$ through a binding (unbinding) event at
site $i$.  

The probability that site $s_i$ is bound in the limit
of infinite time is given by:

\begin{equation}
\label{eq:marginal}
 \sum_{x\in X}P^*(x)[x_i = 1],
\end{equation}
where $P^*$ is the stationary distribution associated with Eq. \ref{eq:cme}.  

Given a vector $\mathbf{k}$ of forward binding constants, define
$\mathbf{k}^x$ as $\prod_{i=1}^Gk_i^{x_i}$.  If
$y\overset{i}\rightarrow x$, then the condition of detailed balance
implies that the equation:

\begin{equation}
\label{eq:detailed_balance}
P^*(y) f(y)k^+_i = P^*(x)
\end{equation}
holds for each pair of related states $y$ and $x$.  Define for convenience:

\begin{equation}
  \label{eq:q}
  Q(x) = (k^{\b{${|x|}$}})\mathbf{k}^x
\end{equation}
where $n^{\b{$m$}}$ is the falling factorial in Knuth notation.
Together, the equations described in \ref{eq:detailed_balance} imply:

\begin{equation}
  \label{eq:stationary_probability}
  P^*(x)=\frac{Q(x)}{\sum_{x'\in X}Q(x')}.
\end{equation}

Finally, combining Eqs. \ref{eq:stationary_probability}
and \ref{eq:marginal}, we obtain:
\begin{equation}
  \label{eq:stochastic_occupancy}
  P(s_i) = \frac{\sum_{x\in X}Q(x)[x_i = 1]}{\sum_{x\in X}Q(x)}
\end{equation}
by marginalizing over all states in which $s_i$ is bound.
% P^*(x)=\frac{\sum_{x\in X |  (k)_{\sum x}\mathbf{k}^x}

\subsection{Derivation of the chemical potential}

In the last section we saw that the occupancy for an arbitrary site
could be computed exactly from the chemical master equation.  It is
not obvious, however, what relationship this expression bears to the
deterministic form found in Eq. \ref{eq:s_fd}.  In this section we
explore this question.

Instead of writing $P(s_i)$ explicitly as:

\begin{equation}
  \label{eq:explicit}
  P(s_i) = \frac{k_ip + k_ik_1p(p-1) + k_ik_2p(p-1) + \ldots + k_ik_Gp(p-1) + \ldots + (k_ik_1k_2\ldots k_{(p-1)})p! + \ldots (k_ik_{G-p+2}k_{G-p+3}\ldots k_{G})p!}{1 + k_1p + k_2p + \ldots k_Gp + \ldots + (k_1k_2\ldots k_{p})p! + \ldots (k_{G-p+1}k_{G-p+2}\ldots k_{G})p!}
\end{equation}
let us write it as:
\begin{align*}
\label{eq:implicit}
P(s_i) =& \frac{\{s_i\ \mathrm{terms}\}}{\{\mathrm{all\ terms} \}}\\
P(s_i) =& \frac{\{s_i\ \mathrm{terms}\}}{\{s_i\ \mathrm{terms}\} + \{\mathrm{non}\text{--}s_i\ \mathrm{terms}\}}\\
P(s_i) =& \frac{1}{1 + \frac{\{\mathrm{non}\text{--}s_i\ \mathrm{terms}\}}{\{s_i\ \mathrm{terms}\}}}.
\end{align*}

Now each term either contains 


In the next section we explore the model's behavior for
various copy numbers and genomic background settings.
\section{Data and Methods}
Experiments concerning \textit{E. coli} TFs were conducted on the
\textit{Escherichia coli} K12 MG1655 genome \cite{blattner97} or on
randomized control genomes of equivalent length with uniform base
probabilities.  Binding energy matrices were derived via the TRAP
binding model \cite{roider07} from every \textit{E. coli}
transcription factor in the PRODORIC database \cite{prodoric03}
(release 8.9) having ten or more experimentally verified sites.

Analysis of the Chip-Seq experiment performed in \cite{smollett12} was
conducted with the \textit{Mycobacterium tuberculosis} H37Rv genome.
Since the model used in \cite{smollett12} was strain 1424, derived
from H37Rv, we checked whether the reported sites were conserved in
strain H37Rv.  All sites were conserved except Site 1; we therefore
analyzed a synthetic genome derived from the H37Rv genome by replacing
the strongest LexA site in the corresponding binding region with Site
1.

All genomic data was obtained from NCBI via FTP.

All code was implemented in Python and is available at
\texttt{github.com/poneill/chemical\_potential}.  In particular, the
relationship between copy number and chemical potential for a given
genomic binding energy landscape was determined by summing
\ref{eq:s_fd} over all $\epsilon_i$ for a given value of $\mu$, which
requires approximately four seconds of wall-clock time on a moderately
endowed desktop computer.  To compute the value of $\mu$ corresponding
to a given $k$, the previous technique generalizes to a bisection
algorithm.

\section{Results and Discussion}
\subsection{$\mu-k$ diagrams}
We are interested to know how well the prevailing approximation in the
literature, given in Eq. \ref{eq:mu_approx}, agrees with the exact
solution.  In Figure \ref{fig:LexA_mu_vs_k}, the chemical potential is
computed both exactly and approximately as a function of copy number,
using both the \textit{E. coli} genome and a random control genome of
equivalent length and mononucleotide frequencies.  Several points
deserve mention.  Firstly, although we witness good agreement between
$\mu$ and $\hat\mu$ in the limit of low copy number on both the real
and synthetic genomes, the curves diverge as copy number increases.
Moreover, $\mu$ increases more rapidly than $\hat\mu$, in agreement
with the phenomenological remark made earlier about
Eq. \ref{eq:p_vs_k}: since $\mu$ is related to $p$ by Eq.
\ref{eq:mu_lnp}, we can verify the slope of the $\mu$ vs. $\log k$
curve to be approximately 1 for large $k$ (rather than $\frac{1}{Z}$
for $\hat\mu$).  Lastly, it is evident that for low copy number, the
chemical potential of the \textit{E. coli} genome is much lower than
that of a random control, regardless of the method of solution.  This
discrepancy is attributable to the enrichment of strong binding sites
in the \textit{E. coli} genome which trap TF copies, lowering its
effective concentration.  As copy number grows large, however, the
strong binding sites become saturated and the chemical potential curve
becomes indistinguishable from that of the control genome.  This
result is in agreement with the well-known finding that
\textit{E. coli} binding energy distribution for a given TF is only
distinguishable from a random nucleotide model in its
over-representation of strong sites\cite{gerland02}\cite{lassig07}; we
find it striking that the cell regulates TF concentration at
approximately the copy number at which the chemical potential becomes
indistinguishable from that of a random control.  The curves for
$\hat\mu$ on the real and control genomes, on the other hand, differ
only in their values of the partition function and therefore cannot
capture this effect.  These results are not specific to LexA: they can
be shown in a variety of transcription factors of \textit{E. coli}, as
evidenced in Fig. \ref{fig:mu_vs_k_examples}.

Approximation of the chemical potential can lead to considerable
divergence from the exact solution in the range of biologically
plausible copy numbers.  How significant is an approximation error of
on the order of $\sim 5k_BT$, as in the case of LexA?


\begin{figure}[ht]
  \centering
  \includegraphics[scale=0.8]{../../results/fig/mu_k_figs/LexA_mu_k_fig.png}
  \caption{Chemical potential for LexA as a function of copy number.
    The exact solution ($\mu$) is compared to the approximation($\hat\mu$) on both the
    \textit{E. coli} genome and a random control genome.}
  \label{fig:LexA_mu_vs_k}
\end{figure}



\begin{figure}[ht]
  \centering
  \subfigure{\includegraphics[scale=0.3]{../../results/fig/mu_k_figs/Crp_mu_k_fig.png}}
  \subfigure{\includegraphics[scale=0.3]{../../results/fig/mu_k_figs/FliA_mu_k_fig.png}}
\subfigure{\includegraphics[scale=0.3]{../../results/fig/mu_k_figs/Fur_mu_k_fig.png}}
\subfigure{\includegraphics[scale=0.3]{../../results/fig/mu_k_figs/OxyR_mu_k_fig.png}}
  \caption{$\mu$-$k$ diagrams for a variety of \textit{E. coli} TFs}
  \label{fig:mu_vs_k_examples}
\end{figure}

\subsection{Regulon Occupancy Predictions}
In this section we compute the equilibrium occupancy of a TF's
reported binding sites given $\mu(k)$ and $\hat\mu(k)$, assuming ten
copies of the TF for each true binding site.

  \begin{figure}[ht]
    \centering
\includegraphics[scale=0.3]{../../results/fig/mu_mu_hat_figs/LexA_mu_mu_hat_fig.png}
    \caption{Binding probabilities for LexA binding sites given exact solution ($\mu$) and approximate solution ($\hat\mu$)}
    \label{fig:LexA_site_killing_exp}
  \end{figure}

  In Fig. \ref{fig:LexA_site_killing_exp}, we observe that use of the
  approximated chemical potential leads to significant underestimation
  of binding probabilities, with many binding sites predicted bound
  only under the exact chemical potential.  Although
  Fig. \ref{fig:LexA_site_killing_exp} compares $\mu$ and $\hat\mu$ at
  a single, physiologically plausible copy number, it is also
  instructive to consider how the total probability mass assigned to
  the verified binding sites varies with copy number, allowing us to
  compare $\mu$ and $\hat\mu$ at all concentrations simultaneously and
  determine the concentration at which the collection becomes saturated.

  % If $mu(k)$ and $\hat\mu(k)$ yield equivalent predictions, then the
  % data in the binding site plot will conform to the straight line
  % $P(s|\mu(k)) = P(s|\hat\mu(k))$.  Judging from Fig.
  % \ref{fig:LexA_site_killing_exp}, however, the use of $\hat\mu$ leads
  % to significant under-estimation of binding.  Indeed, many points
  % occupy the lower-right corner of the plot, denoting a bound
  % probability $\sim 1$ according to the exact method, but $\sim 0$
  % according to the approximation.  (The site occupying the bottom-left
  % corner belongs to the \textit{rpsU-dnaG-rpoD} regulon.)


  % In the last section it was shown that $\hat\mu$ can lead to
  % significant underestimations of the occupancies of true binding
  % sites.  

  % In this section we consider the relationship between copy number and
  % total occupancy, summed over the collection of binding sites.  

  This relationship is visualized in Fig.
  \ref{fig:occupancy_vs_copy_number}.  Two related points are
  immediately evident.  Firstly, the binding response is more
  sensitive to change in copy number when the chemical potential is
  computed exactly.  Relatedly, the idealized chemical potential
  predicts that the collection of functional binding sites will not
  saturate until copy number is far in excess of physiological values;
  the LexA repressor is not predicted to fully occupy its known sites
  until copy number exceeds $10^6$.  Additional examples are provided
  in Fig. \ref{fig:occupancy_vs_copy_number_examples}, illustrating
  the same effect.

  \begin{figure}[ht]
    \centering
    \includegraphics[scale=0.3]{../../results/fig/k_occupancy_figs/LexA_k_occupancy_fig.png}
    \caption{Aggregate binding site occupancy as a function of copy number}
    \label{fig:occupancy_vs_copy_number}
  \end{figure}

  \begin{figure}[ht]
  \centering
\subfigure{\includegraphics[scale=0.3]{../../results/fig/k_occupancy_figs/Crp_k_occupancy_fig.png}}
\subfigure{\includegraphics[scale=0.3]{../../results/fig/k_occupancy_figs/FliA_k_occupancy_fig.png}}
\subfigure{\includegraphics[scale=0.3]{../../results/fig/k_occupancy_figs/Fur_k_occupancy_fig.png}}
\subfigure{\includegraphics[scale=0.3]{../../results/fig/k_occupancy_figs/OxyR_k_occupancy_fig.png}}
  \caption{Occupancy/copy number diagrams for a variety of \textit{E. coli} TFs}
  \label{fig:occupancy_vs_copy_number_examples}
\end{figure}

\subsection{Comparison with Chip-chip data}
Finally, we compare the occupancy predictions of the exact and
idealized chemical potential against a \textit{Mycobacterium
  tuberculosis} LexA experiment performed by Smollett \textit{et al.}
\cite{smollett12} Average peak intensities for 25 reported binding
sites were compared to predicted occupancy for $\mu$ and $\hat\mu$ and
Pearson correlation coefficients were computed at various copy
numbers.

Results are shown in Fig. % \ref{fig:smollett_chip_seq}.
When the
chemical potential is computed exactly, correlation is maximized for
copy number in the physiological range, whereas use of the
approximated chemical potential leads to predictions which lag by more
than two orders of magnitude.  The correlation peak also is much
sharper given the exact solution, in accordance with the previous
result.


% \begin{figure}[ht]
%   \centering
%   \includegraphics[scale=0.8]{results/smollett_figs/smollett_fig.png}
%   \caption{Correlation of LexA Chip-Seq binding intensities with occupancy predictions for exact and approximated chemical potentials}
%   \label{fig:smollett_chip_seq}
% \end{figure}

\subsection{Summary}
The use of the ideal gas approximation for the computation of the
chemical potential of a transcription factor can lead to erroneous
conclusions.  Firstly, it obscures a qualitative behavior exhibited by
the binding distributions of real genomes but lacked by those of
randomized controls: stronger-than-expected functional sites create
energy wells which, once saturated, lead to sharp increases in the
chemical potential as a function of copy number.  In other words, the
exact chemical potentials of real and synthetic genomes become
indistinguishable in the limit of large copy number, whereas the
approximate chemical potentials remain distinct.  Secondly, use of the
approximate chemical potential can lead to significant
under-prediction of binding to functional sites; as much as 70\% of
the binding sites of several regulons were predicted unbound even when
copy number exceeded site number by an order of magnitude.  Thirdly,
the approximated chemical potential not only predicts less binding,
but a slower functional response to the marginal increase in copy
number.  This effect, observed recently in the context of Hill
coefficients of individual sites \cite{sheinman12}, generalizes to the
regulon as a whole as evidenced in
Figs. \label{fig:occupancy_vs_copy_number}
and \label{fig:occupancy_vs_copy_number_examples}.  Lastly, comparison
of the model to available Chip-Seq data shows qualitatively better
agreement with an exact rather than approximated value.



 \newpage
\bibliography{bib/bibliography}{}
\bibliographystyle{abbrv} \newpage
\end{document}
